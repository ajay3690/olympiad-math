\begin{enumerate}
    \item Let $n$ be a positive integer such that $1 \leq n \leq 1000$. Let $M_n$ be the number of integers in the set 
	    $X_n =  \cbrak{ \sqrt{4n+1}, \sqrt{4n+2}, \ldots, \sqrt{4n+1000} }$.
		Let 
		\begin{align}
a = \max{M_n : 1 \leq n \leq 1000},
		\end{align}
	and	\begin{align}
b = \min{M_n : 1 \leq n \leq 1000}.		
		\end{align}
		\\Find $a - b$.\hfill(IOQM 2015)
    
    \item Find the number of elements in the set
	    \begin{align}
   \brak{a, b} \in \cbrak{N} : 2 \leq a, b \leq 2023, \log_a \brak{b} + 6 \log_b \brak{a} = 5.
	    \end{align}\hfill(IOQM 2015)


    \item Let $\alpha$ and $\beta$ be positive integers such that 

	    \begin{align}
\frac{16}{37} < \frac{\alpha}{\beta} < \frac{7}{16}.
	    \end{align}
 Find the smallest possible value of $\beta$.\hfill(IOQM 2015)
    
    \item For $n \in N$ , let $P\brak{n}$ denote the product of the digits in $n$ and $S\brak{n}$ denote the sum of the digits in $n$ . Consider the set 
	    \begin {align}
		A =  \cbrak{ n \in N : P\brak{n}  is  non-zero, square  free  and  S\brak{n}  is  a  proper  divisor   of   P\brak{n} }.
		\end {align}
Find the maximum possible number of digits of the numbers in $A$ .\hfill(IOQM 2015)
    
    \item For any finite non-empty set $X$ of integers, let $\max\brak{X}$ denote the largest element of $X$ and $|X|$ denote the number of elements in $X$. If $N$ is the number of ordered pairs $\brak{A, B}$ of finite non-empty sets of positive integers, such that
	    \begin{align}
\max(A) \times |B| = 12 \quad \text{and}
	    \end{align}
		\begin{align}
			\quad |A| \times \max\brak{B} = 11,
		\end{align}
 and $N$ can be written as $100a + b$ where $a, b$ are positive integers less than 100, find $a + b$.\hfill(IOQM 2015)
 \item The sequence $\langle a_n \rangle_{n \geq 0}$ is defined by $a_0 = 1$, $a_1 = -4$, and $a_{n+2} = -4a_{n+1} - 7a_n$ for $n \geq 0$. Find the number of positive integer divisors of $a_{250} - a_{49} a_{51}$.\hfill(IOQM 2015)
    
    \item A quadruple $\brak{a, b, c, d}$ of distinct integers is said to be balanced if $a +b = c + d$ and $a < b < c < d$. Find the number of balanced quadruples of distinct integers in the set $\cbrak{1, 2, \cdots, 12}$.\hfill(IOQM 2015)
     \item There is an integer n\textgreater1. There are n2 stations on a slope of a mountain, all at
different altitudes. Each of two cable car companies, A and B, operates k cable cars; each cable
car provides a transfer from one of the stations to a higher one (with no intermediate stops). The
k cable cars of A have k different starting points and k different finishing points, and a cable car
which starts higher also finishes higher. The same conditions hold for B. We say that two stations
are linked by a company if one can star using
one or more cars of that company (no other movements between stations are allowed).
Determine the smallest positit from the lower station and reach the higher one byve integer k for which one can guarantee that there are two stations
that are linked by both companies.
\hfill(IMO 2020)
\item Find the smallest positive integer $ k $ such that 
$ k(3^3 + 4^3 + 5^3) = a^n $
		for some positive integers $ a $ and $ n $, with $ n > 17 $.\hfill(Prermo 2013)

\item Let $ S(M) $ denote the sum of the digits of a positive integer $ M $ written in base 10. Let $ N $ be the smallest positive integer such that $ S(N) = 2013 $. What is the value of $ S(5N + 2013) $?\hfill(Prermo 2013)

\item Let $ m $ be the smallest odd positive integer for which
$ 1 + 2 + \cdots + m $
is a square of an integer and let $ n $ be the smallest even positive integer for which
$ 1 + 2 + \cdots + n $
is a square of an integer. What is the value of $ m + n $?\hfill(Prermo 2013)

\item What is the maximum possible value of $ k $ for which 2013 can be written as a sum of $ k $ consecutive positive integers?\hfill(Prermo 2013)
\item Let $a,b$ and $c$ be positive integers, no two of which have a common divisor grater than $1$. Show that $2abc-ab-bc-ca$ is the largest integer which cannot be expressed in the form $xbc+yca+zab$, where $x,y$ and $z$ are non-negative integers.\hfill(IMO 1983)

\item Is it possible to choose $1983$ distinct positive integers, all less than or equal to $10^5$, no three of which are consecutive terms of
an arithmetic progression ? justify your answer.\hfill(IMO 1983)

\item Find one pair of positive integers $a$ and $b$ such that :
	$\brak{i}$ $ab\brak{a+b}$ is not divisible by $7$;
	$\brak{ii}$$\brak{a+b}^7-a^7-b^7$ is divisible by $7^7$\hfill(IMO 1984)


\item Let $a,b,c$ and $d$ be odd integers such that  $0<a<b<c<d$ and $ad=bc$. Prove that if $a+d=2^k$ and $b+c=2^m$ for some integers $k$ and   $m$, then $a=1$\hfill(IMO 1984)

\item Let  $n$ and $k$ be given relatively prime natural numbers $k<n$.Each number in the set $M={1,2,...n-1}$ is colored either blue or white. It is given that
	$\brak{i}$ for each $i  \epsilon   M$, both $i$ and $n-i$ have the same color;
	$\brak{ii}$ for each $i \epsilon  M$, $i\neq k$, both $i$ and $\mydet{i-k}$ have the same color. Prove that all numbers in $M$ must have the same color.\hfill(IMO 1985)

\item Given a set $M$ of $1985$ distinct positive integers, none of which has a prime divisor grater than $26$. Prove that $M$ contains at least one subset of four distinct elements whose product is the fourth power of an integer.\hfill(IMO 1985)


\item For every real number $x_1$, construct the sequence $x_1, x_2, ..116  . $by setting 
	\begin{align*} x_{n+1}=x_n\brak{x_n+\frac{1}{4}}\end{align*} for each $n \geq 1$ Prove that there exists exactly one value of $x_1$ for which  \begin{align*} 
0 < x_n<x_{n+1}<1 \end{align*} for every $n$.\hfill(IMO 1985)
\item Let $1 \leq r \leq n$ and consider all subsets of $r$ elements of the set $\cbrak{1,2,..., n}$. Each of these subsets has a smallest  member. Let $F\brak{n,r}$ denote the arithmetic mean of these smallest numbers; prove that $F\brak{n,r}= \frac{n+1}{r+1}$ \hfill(IMO 1981)


 \item \brak{a} For which values of $n > 2$ is there a set of $n$ consecutive positive integers such that     the largest number in the set is a divisor of the least common multiple of the remaining $n-1$ numbers
	 \brak{b}For which values of $n > 2$ is there exactly one set having the stated property?\hfill(IMO 1981)
\item . The function $f\brak{n}$ is defined for all positive integers $n$ and takes on non-negative integer values. Also, for all $m,n$
  \begin{align*}f\brak{m + n} - f\brak{m} - f\brak{n} = 0  \brak{or} 1 \end{align*}
 \begin{align*}f\brak{2} = 0  , f\brak{3} > 0,and  f\brak{9999} = 3333.\end{align*}
       Determine $f\brak{1982}.$ \hfill(IMO 1982)
\item Prove that if $n$ is a positive integer such that the equation. \begin{align*}x^3 - 3xy^2 + y^3 =
n \end{align*}  has a solution in integers $\brak{x, y}$, then it has at least three such solutions. Sh
w that the equation has no solutions in integers when $n = 2891.$ \hfill(IMO 1982)
		\subsection*{ALGEBRA}
 \item Determine the maximum value of $m^{3}+n^{3}$, where $m$ and $n$ are integers satisfying $m, n  \epsilon  \cbrak{1,2,..., 1981}$ and $\brak{n^{2}-mn-m^{2}}^{2}=1$ \hfill(IMO 1981)
\item The function $f\brak{x, y}$ satisfies
 $\brak{1} f\brak{0, y} = y + 1,$ 
 $\brak{2} f\brak{x + 1, 0} = f\brak{x, 1},$ 
  $\brak{3} f(x + 1, y + 1) = f\brak{x, f\brak{x + 1, y}},$ 
  for all non-negative integers $x, y$. Determine $ f\brak{4,1981}$.\hfill(IMO 1981)
		\subsection*{MATHEMATICAL ANALYSIS}
	\item Consider the infinite sequences $\cbrak{x_n}$ of positive real numbers with following properties:     
             $ x_{0}=1,$ and for  all  $i \geq 0, x_{i+1} \leq x_i.$ 
 \brak{a} Prove that for every such sequence, there is $n \geq 1$ such that
                      \begin{align*} \frac{x^{2}_{0}}{x_{1}}+ \frac{x^{2}_{1}}{x_{2}}+ ...+\frac{x^{2}_{n- 1}}{x_{n}} \geq 3.999.\end{align*}
 \brak{b} Find such a sequence for which
                    \begin{align*} \frac{x^{2}_{0}}{x_{1}}+ \frac{x^{2}_{1}}{x_{2}}+ ...+\frac{x^{2        }_{n1}}{x_{n}}< 4.\end{align*} \hfill(IMO 1982)
                    \item  Let $d$ be any positive integer not equal to $2$, $5$, or $13$. Show that one can find distinct $a$, $b$ in the set $\cbrak{2, 5, 13. d}$ such that $ab-1$ is not a perfect square.\hfill(IMO 1986)

\item Let $p_n \brak{k}$ be the number of permutations of the set $\cbrak{1,\dots
	    ,n}$, $n\geq1$,which have exactly $k$ fixed points.Prove that 
		                 \begin{align*}   \sum_{k=0}^{n} k \cdot p_n\brak{k} = n
					             \end{align*}
(Remark:A permtation $f$ of a set $S$ is one-to-one mapping of $S$ onto itself.An element $i$ in $S$ is called a fixed point of the the permutation $f$ if f\brak{i}=i. ) \hfill(IMO 1987)

\item Let $n$ be a positive integer and let $A_1, A_2, \dots, A_{2n+1}$ be subsets o
    f a set $B$. Suppose that 
                 $\brak{a}$ Each $A_i$ has exactly $2n$ elements,
                  $\brak{b}$ Each $A_i \cap A_j \brak{1\leq i \leq j\leq 2n+1}$contains exactl
    y one element, and \\
                  $\brak{c}$ Every element of $B$ belongs to at least two of the $A_i$.
                  
                 For which values of $n$ can one assign to every element of $B$ one of the numbers $0$ and $1$ in such a way that $A_i$ has $0$ assigned to exactly $n$ of its elements?\hfill(IMO 1988)
\item Let $a$ and $b$ be positive integers such that $ab + 1$ divides $a^2 + b^2$. Show that 
               \begin{align*} \frac{a^2+b^2}{ab+1} \end{align*} is the square of an integer.\hfill(IMO 1988)
               \item problem 1 Prove that for any pair of positive integers $k$ and $n$, there exist $k$ positive integers $m_1,m_2,m_3,\ldots$
 (not necessarily different) such that
\begin{align}
	1+\frac{2^{k}-1}{n}=\brak{1+\frac{1}{m_1}}\brak{1+\frac{1}{m_2}}\ldots\brak{1+\frac{1}{m_k}}
\end{align}  \hfill(Imo 2013)
 \item problem2 let $a_{0<} a_{1<} a_{2 <} \ldots$ be an infinite sequence of positive integers.prove that there exists a unique integer $n    \geq 1$such that          \begin{align}                                                                                                                                                                             a_{n< }\frac{a_0+a1+\ldots+a_n}{n} < a_{n+1}.                                                                                                                \end{align} \hfill(Imo2014) \hfill(Imo 2014)   
\item Problem 3. For each positive integer $n,$ the Bank of Cape Town ienes coins of denomination$\frac{1}{n}$ Given a finite collection of such coins (of  not  necessarily  differ
    ent  denominations) with total value at most $99 +\frac{1}{2}$ prove that it is possible to split this collection into $100$ or fewer groups, such that each group has total value at most $1$. \hfill(Imo2014)
    \item Prove that for each positive integer $n$ there exist $n$ consecutive positive integers none of which is an integral power of a prime number. \hfill(IMO 1989)

	\item A permutation $\brak{x_1,x_2,....,x_m}$ of the set \{1,2.....,2n\}. where $a$ is a positive integer, is said to have property $P$ if $\mydet{x_i - x_{i+1}} = n $ for at least one in\{1,2,....,2n-1\}. Show that, for each $n$, there are more permitations with property $P$ than without.\hfill(IMO 1989)

	\item Determine all integers $n>1$  such that
		\begin{align*} \frac{{2^n}+1}{n^2}\end{align*}is integer.\hfill(IMO 1990)


 \item Given a triangle $ABC$, let $I$ be the center of its inscribed circle. The internal bisectors of the angles $A, B, C$ meet the opposite sides in $A', B', C'$ respectively. Prove that
	 \begin{align*}\frac{1}{4} < \frac{AI. BI. CI.}{AA'. BB'. CC'.}\leq\frac{8}{27}\end{align*}. \hfill(IMO 1991)


\item Let $n > 6$ be an integer and $a_1, a_2,....,a_k $ be all the natura numbers less than $n$ and relatively prime to $n$ If \begin{align*}
a_2-a_1=a_3-a_2=......=a_k-a_{k-1} > 0,\end{align*}
       prove that $n$ must be either a prime number or a power of 2.\hfill(IMO 1991)
       \item In a finite sequence of real numbers the sum of any seven successive terms is negative, and the sum of any eleven successive terms is positive. Determine the maximum number of terms in the sequence.\hfill(Imo 1977)

\item 	Let $n$ be a given integer \textgreater $2$, and let $V_{n}$ be the set of integers $1+ kn$, where $k = 1, 2 ,\ldots A$ number $m \epsilon V_{n}$ is called indecomposable in $V_{n}$, if there do not exist numbers $p$ ,$q \epsilon V_{n}$ such that $pq = m$. Prove that there exists a number $r \epsilon V_{n}$ that can be expressed as the product of elements indecomposable in $V_{n}$ in more than one way. (products which differ only in the order of their factors will be considered the same).\hfill(Imo 1977)

\item Let $a$ and $b$ be positive integers. When $a^2 + b^2$ is divided by $a+b$, the quotient is $q$ and the remainder is $r$. Find all pairs \brak{a, b} such that $q^2 + r = 1977.$ \hfill(Imo 1977)

\item	Let $f\brak{n}$ be a function defined on the set of all positive integers and having all its values in the same set. Prove that if \begin{align}f\brak{n + 1} \textgreater f\brak{f\brak{n}}\end{align} for each positive integer $n$, then \begin{align}f\brak{n} = n\end{align} for each $n$
		\hfill(Imo 1977)

	\item $m$ and $n$ are natural numbers with $1 \leq m \textless n$ In their decimal representations, the last three digits of $1978$ are equal, respectively, to the last three digits of $1978$". Find $m$ and $n$ such that $m+n$ has its least value.\hfill(Imo 1978)

\item The set of all positive integers is the union of two disjoint subsets 
\begin{align}
{f\brak{1}, f\brak{2} ,\ldots,f\brak{n},\ldots} ,{ g\brak{1},g\brak{2},\ldots,g\brak{n},\ldots} 
\end{align},where
\begin{align}
f\brak{1}\textless f\brak{2} \textless \ldots \textless f\brak{n} \textless \ldots,\\ g\brak{1} \textless g\brak{2} \textless \ldots \textless g\brak{n}\textless \ldots\\, and,\ \   g\brak{n}=f\brak{f\brak{n}} + 1
\end{align}
for all $n \geq 1$
		. and Determine $ƒ\brak{240}.$\hfill(Imo 1978)

\item Let ${a_{k}\brak{k=1,2,3.\ldots,n,\ldots}}$ be a sequece of distinct positive integers. Prove that for all natural numbers $n$,\begin{align}\sum_{k=1}^{n} \frac{a_{k}}{k^2} \geq \sum_{k=1}^{n} \frac{1}{k}\end{align}\hfill(Imo 1978)

\item Let $p$ and $q$ be natural numbers such that \begin{align}\frac{p}{q}=-\frac{1}{2}+\frac{1}{3}-\frac{1}{4}+\ldots -\frac{1}{1318}+\frac{1}{1319}\end{align}.Prove that $p$ is divisible by $1979$.\hfill(Imo 1979)

\item For any positive integer $n$, let d{\brak{n}} denote the number of positive divisors of $n$ (including $1$ and $n$ itself). Determine all positive integers k such that $ \frac{d\brak{n^2}} {d\brak{n}}  = k$ for some $n$.\hfill(IMO 1998) 

\item Determine all pairs \brak{a, b} of positive integers such that $ab^2 + b + 7$ divides $a^2    b + a + b$.\hfill(IMO 1998)

\item  Consider all functions $f$ from the set $N$ of all positive integers into itself satisfying $f\brak{t^2f\brak{s}} = s\brak{f\brak{t}}^2$ for all $s$ and $t$ in $N$. Determine the least possible value of ${f\brak{1998}}$.\hfill(IMO 1998)

\item Determine all pairs \brak{n,p} of positive integers such that$p$ is a prime,$n$ not exceeded $2p$,and$\brak{p-1}^n+1$ is divisible by $n^{p-1}$.\hfill(IMO 1999)

\item Can we find $N$ divisible by just $2000$ different primes, so that $N$ divides $2^N + 1$? [$N$ may be divisible by a prime power.]\hfill(IMO 2000)    
\item Let $ABC$ be a triangle with incentre I.A point P in the interior of the triangle satisfies
\begin{align*}
\angle{PBA} + \angle{PCA} = \angle{PBC}+ \angle{PCB} \end{align*} 
Show that $AP\geq{AI}$,and that equality holds if only if $P=I$.\hfill(IMO2006)
\item Determine all pairs (x,y) of integers such that
 \begin{align*}
 1+2^{x}+2^{x+1}=y^{2}
 \end{align*}\hfill(IMO 2006)
 \item Let $N$ be the set of positive integers. Determine all functions $g:N \rightarrow N $ such that
				\begin{align}
					(g(m)+n) (m+g(n))
				\end{align}
				is a perfect square for all $m,n \in N $.\hfill(IMO2010)
			\item In each of six boxes $B_{1}, B_{2}, B_{3}, B_{4}, B_{5}, B_{6}$ there is initially one coin. There are two types of operation allowed:
				Type $1$: Choose a nonempty box $B_{j}$ with $1\leq{j}\leq{5}$. Remove one coin from $B_{j}$ and add two coins to $B_{j+1}$.
				Type $2$: Choose a nonempty box $B_{k}$ with $1\leq{k}\leq{4}$. Remove one coin from $B_{k}$ and exchange the contents of (possible empty) boxes $B_{k+1}$ and $B_{k+2}$.
				Determine whether there is a finite sequence of such operations that results in boxes $B_{1}$, $B_{2}$, $B_{3}$, $B_{4}$, $B_{5}$ being empty and box $B_{6}$ containing exactly $2010^{2010^{2010}}$ coins. (Note that $a^{(b^{c})}$.)\hfill(IMO2010)
			\item Let $a_{1}, a_{2}, a_{3}$,\dots be a sequence of positive real numbers. Suppose that for some positive integer $s$, we have
				\begin{align}
					a_{n}=max\{a_{k}+a_{n-k}\vert1\leq{k}\leq{n-1}\}
				\end{align}
				for all $n>s$. Prove that there exist positive integers $l$ and $N$, with $l\leq{s}$ and such that $a_{n}=a_{l}+a_{n-l}$ for all $n\leq{N}$.\hfill(IMO2010)
			\item Given any $setA=\{a_{1}, a_{2}, a_{3}, a_{4}\}$ of four distinct positive integers, we denote the sum $a_{1}+a_{2}+a_{3}+a_{4}$ by $s_{A}$. Let $n_{A}$ denote the number of pairs $ ( i, j) $ with $1\leq{i}\leq{j}\leq{4}$ for which $a_{i}+a_{j}$ divides $s_{A}$. Find all sets $A$ of four distinct positive integers which achieve the largest possible value of $n_{A}$.\hfill(IMO2011)
			\item  Let $f$ be a function from the set of integers to the set of positive integers. Suppose that, for any two integers $m$ and $n$, the difference $f\brak m -f\brak n$ is divisible by $f( m-n)$. Prove that, for all integers $m$ and $n$ with $f \brak m \leq{f\brak n}$, the number $f\brak n$ is divisible by $f\brak m$.\hfill(IMO2011)
			\item  Let $n\geq{3}$ be an integer, and let $a_{2}, a_{3},\dots, a_{n}$  be positive real numbers such that $a_{2}a_{3} \dots a_{n}=1$. Prove that
				\begin{align}
					(1+a_{2}) ^{2}  (1+a_{3}) ^{3} \dots (1+a_{n}) ^{n} > n^{n}.\hfill(IMO2012)
				\end{align}
			\item Find all functions $f:Z \rightarrow Z $ such that, for all integers $a, b, c$ that satisfy $a+b+c = 0$, the following equality holds:
				\begin{align}
					f(a)^2+f(b)^2+f(c)^2=2f(a)f(b)+2f(b)f(c)+2f(c)f(a).
				\end{align}
				(Here $Z$ denotes the set of integers.)\hfill(IMO2012)
				\brak a Prove that, for any real numbers $x_{1} \leq x_{2} \leq \dots \leq x_{n}$ 
	 $\{ | x_{i} - a_{i} | : 1 \leq  i \leq n \} \geq$  $\frac {d}{2}$. \brak *                                \brak b Show that there are real numbers $x_{1} \ leq x_{2} \leq \dots \leq x_{n}$ such that equality holds in \brak *.\hfill(IMO 2007)
\item Let $a and b$ be positive integers. Show that if     $4ab-1$ divides $(4a^2-1)^2$, then $a=b$.\hfill(IMO 2007)
\item Let $n$ be a positive integer. Consider           $S={(x,y,z)} : {x,y,z} \epsilon i{0,1,\dot,n}$,     ${x+y+z>0}$  as a set of $(n+1)^{3}-1$ points in three-dimensional space.Determine the smallest possible number of planes, the union of which contains $S$ but doesnot include $(0,0,0)$.\hfill(IMO 2007) 
	\item Prove that                                         7 $\frac{x^2}{{(x-1)}^{2}}$+$\frac{y^2}{{(y-1)}^{2}}$+$\frac{z^2}{{(z-1)}^{2}} \geq 1$
	for all real numbers $x, y, z$, each different from $1$, and satisfying $xyz = 1$.$\brak
{b}$  Prove that equality holds above for infinitely many triples of rational numbers $x, y, z$, each different from  $1$, and satisfying $xyz=1$.\hfill(IMO 2008)
\item Prove that there exist infinitely many po    sitive integers $n$ such that $n^2+1$ has a prime divis    or which is greater than $2n+\sqrt2n$.\hfill(IMO 2008)

\item  Determine all functions $f$ from the set of positive integers to the set of positive integers such that, for all positive integers $a$ and $b$, there exists a non-degenerate triangle with sides of lengths \\$a, f (b)$ and $f (b+f(a)-1).$ \\
                $(A triangle is non-degenerate if its v    ertices are not collinear)$.\hfill(IMO 2009)
		\item Let $a_{1}$, $a_{2}$,\dots, $a_{n}$ be distinct positive integers and let $M$ be a set of $n-1$ positive integers not containing $s = a_{1}+a_{2}$+ \dots + $a_{n}$. A grasshopper is to jump along the real axis, starting at the point 0and making $n$ jumps to the right with lengths $a_{1}, a_{2}\dots,a_{n}$ in some order. Prove that the order can be chosen in such a way  that the grasshopper never lands on any point in $M$.\hfill(IMO 2009)
		\item Find all positive integers $n$ for which each cell of     an $n\times n$ table can be filled with one of the letters \begin{align}I,M and O\end{align} in such a way that:                                    in each row and each column,one third of the entries are $I$,one third are $M$ and one third are $O;$and in any diagonal,if the number of entries on the diagonal is a multiple of three,the$n$ one third of the entries are $I,$ one thirdv are $M$ and one third are $O.$\hfill (IMO 2016)
\item $A$ set of positive integers is called fragrant if it contains at least two elements and each of its elements has a prime factor in common with at least one of the other elements. Let $P(n) = n ^ 2 + n + 1$ What is the least possible value of the positive integer $b$ such that there exists a  non-negative integer a for which the set \begin{align}{P(a + 1), P(a + 2) ,...,P(a+b)}\end{align} is fragrant?\hfill (IMO 2016)
(a) Prove that Geoff can always fulfil his wish if $n$ is odd.                                       (b) Prove that Geoff can never fulfil his wish if $n$ is even.
\item An ordered pair $(x, y)$ of integers is a primitive point if the greatest common divisor of $r$ and $y$     is $1$. Given a finite set $S$ of primitive points, prove that there exist a positive integer $n$ and integers $ao$, $41$, $4$ such that, for each $(x, y)4$ in $S$, we have\hfill (IMO 2017)                           \begin{align}a_0x^n + a_1x^{n-1}y + a_2x^{n-2}y^2 + \cdots + a_{n-1}xy^{n-1} + a_ny^n = 1.\end{align}
		\item Let $a1,a2$,... be an infinite sequence of positive integers. Suppose that there is an integer $N\>1$ such that, for each $n\neq N$, the number $01$ is an integer. Prove that there is a positive integer $M$ such that for all $m1>=M$.\hfill (IMO 2018)
 \item  Find all integers $a,b,c$ with $1 < a < b <c$ such that  \hfill(IMO 1992)                                        
 $(a-1)(b-1)(c-1)$ is a divisor of $abc$ - $1$.
\item For each positive integer $n$, $S(n)$ is defined to be the greatest integer such that, for every positive integer $k$ $\leq$ $S(n)$, $n^{2}$ can be  written as the sum of $k$ positive squares.  \hfill(IMO 1992)       

$(a)$ Prove that $S(n)$ $\leq$ $n^{2}$ - $14$ f    or each $n \geq 4$.                                 

$(b)$ Find an integer $n$ such that $S(n)=n^{2}    -14$.                                               

$(c)$ Prove that there are infintely many integers $n$ such that $S(n) = {n^{2}}-14$.
\item Let $m$ and $n$ be positive integers. Let $a_1, a_2, $\dots$ , a_m$ be distinct elements of $\{1, 2, \dots , n\}$ such that whenever $a_i + a_j \leq n$ for some $i, j, 1 \leq i \leq j \leq m$, there exists $k, 1 \leq k \leq m$, with $a_i + a_j = a_k$. Prove that
	$\frac{a_1+a_2+\dots+a_m}{m} \geq \frac{n+1}{2}$.        \hfill(IMO 1994)
\item Determine all ordered pairs $(m,n)$ of positive integers such that                               
  $\frac{{n^{3}}+1}{mn-1}$
                                                
		 is an integer.  \hfill(IMO 1994)
\end{enumerate}
