\begin{enumerate}
    \item Let $n$ be a positive integer such that $1 \leq n \leq 1000$. Let $M_n$ be the number of integers in the set 
	    $X_n =  \cbrak{ \sqrt{4n+1}, \sqrt{4n+2}, \ldots, \sqrt{4n+1000} }$.
		Let 
		\begin{align}
a = \max{M_n : 1 \leq n \leq 1000},
		\end{align}
	and	\begin{align}
b = \min{M_n : 1 \leq n \leq 1000}.		
		\end{align}
		\\Find $a - b$.\hfill(IOQM 2015)
    
    \item Find the number of elements in the set
	    \begin{align}
   \brak{a, b} \in \cbrak{N} : 2 \leq a, b \leq 2023, \log_a \brak{b} + 6 \log_b \brak{a} = 5.
	    \end{align}\hfill(IOQM 2015)


    \item Let $\alpha$ and $\beta$ be positive integers such that 

	    \begin{align}
\frac{16}{37} < \frac{\alpha}{\beta} < \frac{7}{16}.
	    \end{align}
 Find the smallest possible value of $\beta$.\hfill(IOQM 2015)
    
    \item For $n \in N$ , let $P\brak{n}$ denote the product of the digits in $n$ and $S\brak{n}$ denote the sum of the digits in $n$ . Consider the set 
	    \begin {align}
		A =  \cbrak{ n \in N : P\brak{n}  is  non-zero, square  free  and  S\brak{n}  is  a  proper  divisor   of   P\brak{n} }.
		\end {align}
Find the maximum possible number of digits of the numbers in $A$ .\hfill(IOQM 2015)
    
    \item For any finite non-empty set $X$ of integers, let $\max\brak{X}$ denote the largest element of $X$ and $|X|$ denote the number of elements in $X$. If $N$ is the number of ordered pairs $\brak{A, B}$ of finite non-empty sets of positive integers, such that
	    \begin{align}
\max(A) \times |B| = 12 \quad \text{and}
	    \end{align}
		\begin{align}
			\quad |A| \times \max\brak{B} = 11,
		\end{align}
 and $N$ can be written as $100a + b$ where $a, b$ are positive integers less than 100, find $a + b$.\hfill(IOQM 2015)
 \item The sequence $\langle a_n \rangle_{n \geq 0}$ is defined by $a_0 = 1$, $a_1 = -4$, and $a_{n+2} = -4a_{n+1} - 7a_n$ for $n \geq 0$. Find the number of positive integer divisors of $a_{250} - a_{49} a_{51}$.\hfill(IOQM 2015)
    
    \item A quadruple $\brak{a, b, c, d}$ of distinct integers is said to be balanced if $a +b = c + d$ and $a < b < c < d$. Find the number of balanced quadruples of distinct integers in the set $\cbrak{1, 2, \cdots, 12}$.\hfill(IOQM 2015)
     \item There is an integer n\textgreater1. There are n2 stations on a slope of a mountain, all at
different altitudes. Each of two cable car companies, A and B, operates k cable cars; each cable
car provides a transfer from one of the stations to a higher one (with no intermediate stops). The
k cable cars of A have k different starting points and k different finishing points, and a cable car
which starts higher also finishes higher. The same conditions hold for B. We say that two stations
are linked by a company if one can star using
one or more cars of that company (no other movements between stations are allowed).
Determine the smallest positit from the lower station and reach the higher one byve integer k for which one can guarantee that there are two stations
that are linked by both companies.
\hfill(IMO 2020)
\item Find the smallest positive integer $ k $ such that 
$ k(3^3 + 4^3 + 5^3) = a^n $
		for some positive integers $ a $ and $ n $, with $ n > 17 $.\hfill(Prermo 2013)

\item Let $ S(M) $ denote the sum of the digits of a positive integer $ M $ written in base 10. Let $ N $ be the smallest positive integer such that $ S(N) = 2013 $. What is the value of $ S(5N + 2013) $?\hfill(Prermo 2013)

\item Let $ m $ be the smallest odd positive integer for which
$ 1 + 2 + \cdots + m $
is a square of an integer and let $ n $ be the smallest even positive integer for which
$ 1 + 2 + \cdots + n $
is a square of an integer. What is the value of $ m + n $?\hfill(Prermo 2013)

\item What is the maximum possible value of $ k $ for which 2013 can be written as a sum of $ k $ consecutive positive integers?\hfill(Prermo 2013)
\item Let $a,b$ and $c$ be positive integers, no two of which have a common divisor grater than $1$. Show that $2abc-ab-bc-ca$ is the largest integer which cannot be expressed in the form $xbc+yca+zab$, where $x,y$ and $z$ are non-negative integers.\hfill(IMO 1983)

\item Is it possible to choose $1983$ distinct positive integers, all less than or equal to $10^5$, no three of which are consecutive terms of
an arithmetic progression ? justify your answer.\hfill(IMO 1983)

\item Find one pair of positive integers $a$ and $b$ such that :
	$\brak{i}$ $ab\brak{a+b}$ is not divisible by $7$;
	$\brak{ii}$$\brak{a+b}^7-a^7-b^7$ is divisible by $7^7$\hfill(IMO 1984)


\item Let $a,b,c$ and $d$ be odd integers such that  $0<a<b<c<d$ and $ad=bc$. Prove that if $a+d=2^k$ and $b+c=2^m$ for some integers $k$ and   $m$, then $a=1$\hfill(IMO 1984)

\item Let  $n$ and $k$ be given relatively prime natural numbers $k<n$.Each number in the set $M={1,2,...n-1}$ is colored either blue or white. It is given that
	$\brak{i}$ for each $i  \epsilon   M$, both $i$ and $n-i$ have the same color;
	$\brak{ii}$ for each $i \epsilon  M$, $i\neq k$, both $i$ and $\mydet{i-k}$ have the same color. Prove that all numbers in $M$ must have the same color.\hfill(IMO 1985)

\item Given a set $M$ of $1985$ distinct positive integers, none of which has a prime divisor grater than $26$. Prove that $M$ contains at least one subset of four distinct elements whose product is the fourth power of an integer.\hfill(IMO 1985)


\item For every real number $x_1$, construct the sequence $x_1, x_2, ..116  . $by setting 
	\begin{align*} x_{n+1}=x_n\brak{x_n+\frac{1}{4}}\end{align*} for each $n \geq 1$ Prove that there exists exactly one value of $x_1$ for which  \begin{align*} 
0 < x_n<x_{n+1}<1 \end{align*} for every $n$.\hfill(IMO 1985)
\item Let $1 \leq r \leq n$ and consider all subsets of $r$ elements of the set $\cbrak{1,2,..., n}$. Each of these subsets has a smallest  member. Let $F\brak{n,r}$ denote the arithmetic mean of these smallest numbers; prove that $F\brak{n,r}= \frac{n+1}{r+1}$ \hfill(IMO 1981)


 \item \brak{a} For which values of $n > 2$ is there a set of $n$ consecutive positive integers such that     the largest number in the set is a divisor of the least common multiple of the remaining $n-1$ numbers
	 \brak{b}For which values of $n > 2$ is there exactly one set having the stated property?\hfill(IMO 1981)
\item . The function $f\brak{n}$ is defined for all positive integers $n$ and takes on non-negative integer values. Also, for all $m,n$
  \begin{align*}f\brak{m + n} - f\brak{m} - f\brak{n} = 0  \brak{or} 1 \end{align*}
 \begin{align*}f\brak{2} = 0  , f\brak{3} > 0,and  f\brak{9999} = 3333.\end{align*}
       Determine $f\brak{1982}.$ \hfill(IMO 1982)
\item Prove that if $n$ is a positive integer such that the equation. \begin{align*}x^3 - 3xy^2 + y^3 =
n \end{align*}  has a solution in integers $\brak{x, y}$, then it has at least three such solutions. Sh
w that the equation has no solutions in integers when $n = 2891.$ \hfill(IMO 1982)
		\subsection*{ALGEBRA}
 \item Determine the maximum value of $m^{3}+n^{3}$, where $m$ and $n$ are integers satisfying $m, n  \epsilon  \cbrak{1,2,..., 1981}$ and $\brak{n^{2}-mn-m^{2}}^{2}=1$ \hfill(IMO 1981)
\item The function $f\brak{x, y}$ satisfies
 $\brak{1} f\brak{0, y} = y + 1,$ 
 $\brak{2} f\brak{x + 1, 0} = f\brak{x, 1},$ 
  $\brak{3} f(x + 1, y + 1) = f\brak{x, f\brak{x + 1, y}},$ 
  for all non-negative integers $x, y$. Determine $ f\brak{4,1981}$.\hfill(IMO 1981)
		\subsection*{MATHEMATICAL ANALYSIS}
	\item Consider the infinite sequences $\cbrak{x_n}$ of positive real numbers with following properties:     
             $ x_{0}=1,$ and for  all  $i \geq 0, x_{i+1} \leq x_i.$ 
 \brak{a} Prove that for every such sequence, there is $n \geq 1$ such that
                      \begin{align*} \frac{x^{2}_{0}}{x_{1}}+ \frac{x^{2}_{1}}{x_{2}}+ ...+\frac{x^{2}_{n- 1}}{x_{n}} \geq 3.999.\end{align*}
 \brak{b} Find such a sequence for which
                    \begin{align*} \frac{x^{2}_{0}}{x_{1}}+ \frac{x^{2}_{1}}{x_{2}}+ ...+\frac{x^{2        }_{n1}}{x_{n}}< 4.\end{align*} \hfill(IMO 1982)
                    \item  Let $d$ be any positive integer not equal to $2$, $5$, or $13$. Show that one can find distinct $a$, $b$ in the set $\cbrak{2, 5, 13. d}$ such that $ab-1$ is not a perfect square.\hfill(IMO 1986)

\item Let $p_n \brak{k}$ be the number of permutations of the set $\cbrak{1,\dots
	    ,n}$, $n\geq1$,which have exactly $k$ fixed points.Prove that 
		                 \begin{align*}   \sum_{k=0}^{n} k \cdot p_n\brak{k} = n
					             \end{align*}
(Remark:A permtation $f$ of a set $S$ is one-to-one mapping of $S$ onto itself.An element $i$ in $S$ is called a fixed point of the the permutation $f$ if f\brak{i}=i. ) \hfill(IMO 1987)

\item Let $n$ be a positive integer and let $A_1, A_2, \dots, A_{2n+1}$ be subsets o
    f a set $B$. Suppose that 
                 $\brak{a}$ Each $A_i$ has exactly $2n$ elements,
                  $\brak{b}$ Each $A_i \cap A_j \brak{1\leq i \leq j\leq 2n+1}$contains exactl
    y one element, and \\
                  $\brak{c}$ Every element of $B$ belongs to at least two of the $A_i$.
                  
                 For which values of $n$ can one assign to every element of $B$ one of the numbers $0$ and $1$ in such a way that $A_i$ has $0$ assigned to exactly $n$ of its elements?\hfill(IMO 1988)
\item Let $a$ and $b$ be positive integers such that $ab + 1$ divides $a^2 + b^2$. Show that 
               \begin{align*} \frac{a^2+b^2}{ab+1} \end{align*} is the square of an integer.\hfill(IMO 1988)
               \item problem 1 Prove that for any pair of positive integers $k$ and $n$, there exist $k$ positive integers $m_1,m_2,m_3,\ldots$
 (not necessarily different) such that
\begin{align}
	1+\frac{2^{k}-1}{n}=\brak{1+\frac{1}{m_1}}\brak{1+\frac{1}{m_2}}\ldots\brak{1+\frac{1}{m_k}}
\end{align}  \hfill(Imo 2013)
 \item problem2 let $a_{0<} a_{1<} a_{2 <} \ldots$ be an infinite sequence of positive integers.prove that there exists a unique integer $n    \geq 1$such that          \begin{align}                                                                                                                                                                             a_{n< }\frac{a_0+a1+\ldots+a_n}{n} < a_{n+1}.                                                                                                                \end{align} \hfill(Imo2014) \hfill(Imo 2014)   
\item Problem 3. For each positive integer $n,$ the Bank of Cape Town ienes coins of denomination$\frac{1}{n}$ Given a finite collection of such coins (of  not  necessarily  differ
    ent  denominations) with total value at most $99 +\frac{1}{2}$ prove that it is possible to split this collection into $100$ or fewer groups, such that each group has total value at most $1$. \hfill(Imo2014)
    \item Prove that for each positive integer $n$ there exist $n$ consecutive positive integers none of which is an integral power of a prime number. \hfill(IMO 1989)

	\item A permutation $\brak{x_1,x_2,....,x_m}$ of the set \{1,2.....,2n\}. where $a$ is a positive integer, is said to have property $P$ if $\mydet{x_i - x_{i+1}} = n $ for at least one in\{1,2,....,2n-1\}. Show that, for each $n$, there are more permitations with property $P$ than without.\hfill(IMO 1989)

	\item Determine all integers $n>1$  such that
		\begin{align*} \frac{{2^n}+1}{n^2}\end{align*}is integer.\hfill(IMO 1990)


 \item Given a triangle $ABC$, let $I$ be the center of its inscribed circle. The internal bisectors of the angles $A, B, C$ meet the opposite sides in $A', B', C'$ respectively. Prove that
	 \begin{align*}\frac{1}{4} < \frac{AI. BI. CI.}{AA'. BB'. CC'.}\leq\frac{8}{27}\end{align*}. \hfill(IMO 1991)


\item Let $n > 6$ be an integer and $a_1, a_2,....,a_k $ be all the natura numbers less than $n$ and relatively prime to $n$ If \begin{align*}
a_2-a_1=a_3-a_2=......=a_k-a_{k-1} > 0,\end{align*}
       prove that $n$ must be either a prime number or a power of 2.\hfill(IMO 1991)
       \item In a finite sequence of real numbers the sum of any seven successive terms is negative, and the sum of any eleven successive terms is positive. Determine the maximum number of terms in the sequence.\hfill(Imo 1977)

\item 	Let $n$ be a given integer \textgreater $2$, and let $V_{n}$ be the set of integers $1+ kn$, where $k = 1, 2 ,\ldots A$ number $m \epsilon V_{n}$ is called indecomposable in $V_{n}$, if there do not exist numbers $p$ ,$q \epsilon V_{n}$ such that $pq = m$. Prove that there exists a number $r \epsilon V_{n}$ that can be expressed as the product of elements indecomposable in $V_{n}$ in more than one way. (products which differ only in the order of their factors will be considered the same).\hfill(Imo 1977)

\item Let $a$ and $b$ be positive integers. When $a^2 + b^2$ is divided by $a+b$, the quotient is $q$ and the remainder is $r$. Find all pairs \brak{a, b} such that $q^2 + r = 1977.$ \hfill(Imo 1977)

\item	Let $f\brak{n}$ be a function defined on the set of all positive integers and having all its values in the same set. Prove that if \begin{align}f\brak{n + 1} \textgreater f\brak{f\brak{n}}\end{align} for each positive integer $n$, then \begin{align}f\brak{n} = n\end{align} for each $n$
		\hfill(Imo 1977)

	\item $m$ and $n$ are natural numbers with $1 \leq m \textless n$ In their decimal representations, the last three digits of $1978$ are equal, respectively, to the last three digits of $1978$". Find $m$ and $n$ such that $m+n$ has its least value.\hfill(Imo 1978)

\item The set of all positive integers is the union of two disjoint subsets 
\begin{align}
{f\brak{1}, f\brak{2} ,\ldots,f\brak{n},\ldots} ,{ g\brak{1},g\brak{2},\ldots,g\brak{n},\ldots} 
\end{align},where
\begin{align}
f\brak{1}\textless f\brak{2} \textless \ldots \textless f\brak{n} \textless \ldots,\\ g\brak{1} \textless g\brak{2} \textless \ldots \textless g\brak{n}\textless \ldots\\, and,\ \   g\brak{n}=f\brak{f\brak{n}} + 1
\end{align}
for all $n \geq 1$
		. and Determine $ƒ\brak{240}.$\hfill(Imo 1978)

\item Let ${a_{k}\brak{k=1,2,3.\ldots,n,\ldots}}$ be a sequece of distinct positive integers. Prove that for all natural numbers $n$,\begin{align}\sum_{k=1}^{n} \frac{a_{k}}{k^2} \geq \sum_{k=1}^{n} \frac{1}{k}\end{align}\hfill(Imo 1978)

\item Let $p$ and $q$ be natural numbers such that \begin{align}\frac{p}{q}=-\frac{1}{2}+\frac{1}{3}-\frac{1}{4}+\ldots -\frac{1}{1318}+\frac{1}{1319}\end{align}.Prove that $p$ is divisible by $1979$.\hfill(Imo 1979)


\end{enumerate}
