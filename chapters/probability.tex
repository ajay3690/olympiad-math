\begin{enumerate}
	\item Find the number of triples $\brak{a, b, c}$ of positive integers such that
	\begin{enumerate}
		\item $ab$ is a prime;
		\item $bc$ is a product of two primes;
		\item $abc$ is not divisible by square of any prime and
		\item $abc \leq 30$.
	\end{enumerate}\hfill(IOQM 2015)
	\item We call a positive integer alternating if every two consecutive digits in its decimal representation are of different parity.Find all positive integers $n$ such that $n$ has a multiple which is alternating \hfill(IMO 2004)
	\item Find the maximum value of $x_{0}$ for which there exists a sequence $x_{0},x_{1}\dots x_{1995}$ of positive reals with $x_{0}=x_{1995}$, such that for $i=1,\dots,1995$.
 \begin{align}..
x_{i-1.}+\frac{2}{x_{i-1}}=2x_i+\frac{1}{x_i    }.\hfill(IMO 1995)
  \end{align}
\item Let $p$ be an odd prime number. How many p-element subsets $A$ of $\cbrak{1,2,....2p}$ are there, the sum of whose elements is divisible by $p$?\hfill(IMO1995)
\item Find all pairs $(a,b)$ of integers $a,b \geq 1    $ that satisfy the equation                    
\begin{align}                               
a^{b^2}=b^{a}.\hfill(IMO 1997)               
\end{align}
\item For each positive integer $n$, let $f(n)$ denote the number of ways of representing $n$ as a sum of powers of $2$ with nonnegative integer exponents.  Representations which differ only in the ordering of their of their summands are considered to be the same. For instance, $f(4)=4$, because the number $4$ can be represented in the following four ways;     
\begin{align}                                       4;2 + 2;2 + 1 + 1;1 + 1 + 1 + 1.
  \end{align}
 Prove that, for any integer $n \geq{3}$,
 \begin{align}
2^{n^2/4} < f(2^n)<2^{n^2/2}.\hfill(IMO 1997)
 \end{align}
 \item Let ${x_1,x_2,\dots,x_n}$ be the real numbers  satisfying the conditions
  \begin{align}                                    
\mydet{x_1+x_2+\dots+x_n}=1
 \end{align}                                         and                                               
\begin{align}                                         \mydet{x_i}\leq\frac{n+1}{2}      {i=1,2,\dots,n}. \end{align}                                      
Show that there exists a permutation ${y_1, y_2,\dot    s,y_n}$ of ${x_1, x_2,\dots,x_n}$ such that    
\begin{align}                                     
\mydet{y_1+2y_2+\dots+ny_n}\leq\frac{n+1}{2}.     
\end{align}\hfill(IMO 1997)
\item Let $S$ denote the set of nonnegative integers. Find all functions $f$ from $s$ to itself such that
 \begin{align}
f(m+f(n))=f(f(m))+f(n)
    \forall{m}, n \epsilon S. \hfill(IMO 1996)
 \end{align}
\item Let $p$, $q$, $n$ be three positive integers with $p+q<n$. Let $(x_0, x_1,....,x_n)$ be an $(n+1)$-tuple of integers satisfying the following conditons:
  \begin{enumerate}                                
 \item $x_0=x_n=0$.                       
\item For each $i$ with $1\leq{i}\leq{n}$
either $x _i-x_{i-1}=p$ or $x_i-x_{i-1}=-q$.  
Show that there exist indices $i < j$ with $(i,j) \neq(0,n)$, such that $x_i=x_j$.\hfill(IMO 1996)
  \end{enumerate}	

\end{enumerate}

