\begin{enumerate}
\item Let $N$ be the set of natural numbers. Suppose $f : N \rightarrow N$ is a function satisfying the following conditions:
    \begin{enumerate}
	    \item $f\brak{mn} = f\brak{m} f\brak{n}$,
	    \item $f\brak{m} < f\brak{n}$ if $m < n$,
	    \item $f\brak{2} = 2$.
    \end{enumerate}
	What is the value of $\sum_{k=1}^{20} f\brak{k}$ ?\hfill(PRERMO 2012)
	\item One is given a finite set of points in the plane, each point having integer coordinates. Is it always possible to color some of the points in the set red and the remaining points white in such a way that for any straight line $L$ parallel to either one of the coordinate axes the difference (in absolute value) between the numbers of white point and red points on $L$ is not greater than $1$?\hfill(IMO 1986)
\item Let $n$ be an integer greater than or equal to $2$. Prove that if $k^2+k+n$ is prime for all integers $k$ such that $0\leq k\leq \sqrt{n/3}$, then $k^2+k+n$ is prime for all integers $k$ such that $0\leq k\leq n-2$ \hfill(IMO 1987)
	\item A function $f$ is defined on the positive integers by
                  \begin{align*}
  f\brak{1}=1, f\brak{3}=3, \\
  f\brak{2n}=f\brak{n}, \\
                  f\brak{4n+1}=2f\brak{2n+1}-f\brak{n},\\
                  f\brak{4n+3}=3f\brak{2n+1}-2f\brak{n},\\ \end{align*}
                  for all positive integers n.
                  Determine the number of positive integers $n$, less than or equal to $1988$,for which $f(n) = n$.\hfill(IMO 1988)
                  \item Show that set of real numbers x which satisfy the in equality                                                                \begin{align*}\sum{k=1}^{70}\frac{k}{x-k}\geq \frac{5}{4}\\ \end{align*}                                                          is a union of disjoint intervals, the sum of whose lengths is $1988$\hfill(IMO 1988)
\item Let $Q^+$ be the set of positive rational numbers. Construct a function $ f: Q^+ \rightarrow Q^+$ such that 

			\begin{align*}  f\brak{xf\brak{y}}= \frac{f\brak{x}}{y} \end{align*}\\ for all $x , y$ in $Q^+$.\hfill(IMO 1990)


      \subsection*{COMBINATOMICS}

  \item Let $S = \{1,2,3,......,280\}$. Find the smallest integer $n$ such that each $n-$ element subset of $S$ contains five numbers which are pairwise relatively prime.\hfill(IMO 1991)

	  \subsection*{GRAPH THEORY}

  \item Suppose $G$ is a connected graph with $k$ edges. Prove that it is possible to label the edges $1,2.....k$ in such a way that at each vertex which belongs to two or more edges, the greatest common divisor of the integers labeling those edges is equal to $1$.
	  $[$ A graph consists of a set of points, called vertices, together with a set of edges joining certain pairs of distinct vertices. Each pair of vertices. $u, v$ belongs to at most one edge. The graph $G$ is connected if for cach pair of distinct vertices $x, y$ there is some sequence of vertices   $x=v_0,v_1,v_2,.......,v_m = y$  such that each pair $v_i,v_{i+1}\brak{0\leq i < m}$ is joined by an edge of G$.]$ \hfill(IMO 1991)
\end{enumerate}
