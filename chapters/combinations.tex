\begin{enumerate}
	\item A positive integer $n > 1$ is called beautiful if $n$ can be written in one and only one way as $n = a_{1} + a_{2} + \cdots + a_{k} = a_{1} \cdot a_{2} \cdots a_{k} $for some positive integers $a_{1},a_{2}, \cdots , a_{k}$ , where $k > 1$ and$ a_{1}  \geq a_{2} \geq \cdots \geq a_{k}$ . (For example $6$ is beautiful since$ 6 = 3 \cdot 2 \cdot 1 = 3 + 2 + 1$ , and this is unique. But $8$ is not beautiful since $8 = 4 + 2 + 1 + 1 = 4 \cdot 2 \cdot 1 \cdot 1 $  as well as $8 = 2 + 2 + 2 + 1 + 1 = 2 \cdot 2 \cdot 2 \cdot 1 \cdot 1$ , souniqueness is lost.) Find the largest beautiful number less than $100$.\hfill(IOQM 2015)



	\item For $n \in N$ , consider non-negative integer-valued functions $f$ on $\cbrak{1, 2, \cdots , n}$ satisfying $f\brak{i} \geq f\brak{j}$ for $i > j$ and $\sum_{i=1}^n \brak{i + f\brak{i}} = 2023$ . Choose $n$ such that $\sum_{i=1}^n f\brak{i}$ is the least. How many such functions exist in that case?\hfill(IOQM 2015)



	\item In the land of Binary, the unit of currency is called Ben and currency notes areavailable in denominations $1, 2, 2^2, 2^3, \cdots $ Bens. The rules of the Government of Binary stipulate that one can not use more than two notes of any one denomination in any transaction. For example, one can give a change for $2$ Bens in two ways: $2$ one Ben notes or $1$ two Ben note. For $5$ Ben one can give $1$ one Ben note and $1$ four Ben note or $1$ one Ben note and $2$ two Ben notes. Using $5$ one Ben notes or $3$ one Ben notes and $1$ two Ben notes for a $5$ Ben transaction is prohibited. Find the number of ways in which one can give change for $100$ Bens, following the rules of the Government.\hfill(IOQM 2015)	
	\item Unconventional dice are to be designed such that the six faces are marked with numbers from $1$ to $6$ with $1$ and $2$ appearing on opposite faces. Further, each face is colored either red or yellow with opposite faces always of the same color. Two dice are considered to have the same design if one of them can be rotated to obtain a die that has the same numbers and colors on the corresponding faces as the other one. Find the number of distinct dice that can be designed.\hfill(IOQM 2015)
    
	\item Given a $2 \times 2$ tile and seven dominoes ($2 \times 1$ tile), find the number of ways of tiling a $2 \times 7$ rectangle using some of these tiles.\hfill(IOQM 2015)
    
    \item Consider the set 
	    \begin{align}
		    S = \cbrak{\brak{a, b, c, d, e} : 0 < a < b < c < d < e < 100}
	    \end{align}
\\where $a, b, c, d, e$ are integers. If $D$ is the average value of the fourth element of such a tuple in the set, taken over all the elements of $S$, find the largest integer less than or equal to $D$.\hfill(IOQM 2015)
    
    \item Let $P$ be a convex polygon with $50$ vertices. A set $F$ of diagonals of $P$ is said to be minimally friendly if any diagonal $d \in F$ intersects at most one other diagonal in $F$ at a point interior to $P$. Find the largest possible number of elements in a minimally friendly set $F$.\hfill(IOQM 2015)
    \item Find all pairs $(k, n)$ of positive integers such that
\begin{align}
k! = \brak{2n - 1}\brak{2n - 2}\brak{2n - 4} \cdots \brak{2n - 2n + 1}.
\end{align}
\hfill(IMO 2019)
\item There are 4n pebbles of weights 1,2,3.....,4n.Each pebble is coloured in one of n colours and there are four pebbles of each colour.Show that we can arrange the pebbles into two piles so that the following two conditions are both satisfied:

    The total weights of both piles are the same.
    Each pile contains two pebbles of each colour.
\hfill(IMO 2020)
\item Two squirrles,Bushy and jumpy,have collected 2021 walnuts for the winter .jumpy  numbers the walnuts from 1 through 2021,and digs 2021 little holes in a circular pattern in the ground around their favourite tree.The next morning jumpy notices that bushy had placed one walnut into each hole ,but had paid no attention to the numbering .unhappy,Jumpy decides to reorder the walnuts by performing a sequence of 2021 moves.In the k-th move,jumpy swaps the positions of the two walnuts adjacent to walnut k.Prove that there exists a value of k such that ,on the k-th move,jumpy swaps some walnuts a and b such that a\textless k\textless b.
\hfill(IMO 2021)
\item Twenty-one girls and twenty-one boys took part in a mathematical contest. Each contestant solved a t most six problems. For each girl and each boy, at least one problem was solved by both of them. Prove t hat there was a problem that was solved by at least three girls and at least three boys. \hfill(IMO 2001 )
\item $S$ is the set $\{1,2,3,\ldots, 1000000 \}$. Show that for any subset $A$ of $S$ with $101$ elements we can find $100$ distinct elements $x_{i}$ of $S$, such that the sets $\{a+x_{i} a\in A\}$ are all pairwise disjoint.\hfill(IMO 2003)
\item $S$ is the set of all $\brak{h, k}$ wi th $h$, $k$ non-negative integers such that $h + k \ textless n$. Each element of $S$ is colored red or b lue, so that if \brak{h, k} is red and $h'\leq h,k'\ leq k$, then $\brak{h', k'}$ is also red. $A$ type $ 1$ subset of $S$ has $n$ blue elements with differen t first member and a type $2$ subset of $S$ has $n$ blue elements with different second member. Show tha t there are the same number of type $1$ and type $2$ subsets.\hfill (IMO 2002)
\item  To each vertex of a regular pentagon an integer is assigned in such a way that the sum of all five numbers is positive. If three consecutive vertices are assigned the numbers $x$,$y$,$z$ respectively and $y<0$ then the following operation is allowed: the numbers $x$,$y$,$z$ are replaced by $x+y$,$-y$,$z+y$ respectively. Such an operation is performed repeatedly as long as at least one of the live numbers is negative. Determine whether this procedure necessarily comes to and end after a finite number of steps.\hfill(IMO 1986)
\item One is given a finite set of points in the plane, each point having integer coordinates. Is it always possible to color some of the points in the set red and the remaining points white in such a way that for any straight line $L$ parallel to either one of the coordinate axes the difference (in absolute value) between the numbers of white point and red points on $L$ is not greater than $1$?\hfill(IMO 1986)
\item Let $x_1, x_2, \dots, x_n$ be real numbers satisfying $x^2_1 +x^2_2 +\dots+x^2_n = 1$. Prove that for every integer $k\geq2$ there are integers $a_1, a_2, \dots , a_n$, not all $0$, such that $\mydet{a_i}\leq k-1$ for all $i$ and 
                     \begin{align*} \mydet{a_1x_1+a_2x_2+\dots+a_nx_n} \leq \frac{\brak{k-1}\sqrt{n}}{k^n-1} \end{align*}\hfill(IMO 1987)
	 \item Let $n$ be an integer greater than or equal to $2$. Prove that if $k^2+k+n$ is prime for all integers $k$ such that $0\leq k\leq \sqrt{n/3}$, then $k^2+k+n$ is prime for all integers $k$ such that $0\leq k\leq n-2$ \hfill(IMO 1987)
 \item Problem 4. Let $n \geq  3$ be an integer, and consider a circle with $n+1$ equally spaced points marked on it. Consider all labellings of these points with the numbers $0,1,\    ldots n$ such that each label is used exactly once, two such labellings are considered to be the same if one can be obtained from the other by a rotation of the circle. A labelling     is called beautiful if, for any four labels $a< b<c < d$ with $a+d=b+c,$ the chord joining the points labelled $a$ and $d$ does not intersect the chord joi    ning the points labelled $b$ and $c$                                                                                                                                                                                                                                                                                                                                     Let $M$ be the number of beautiful labellings, and let $N$ be the number of ordered pairs $\brak{x, y}$ of  positive integers such that $x+y\leq n and gcd \brak{x,y} = 1$. Prove that                                                                                                                                                                                                                                                                                                                                $m=n+1$  
 \item An international society has its members from six different countries. The list of members contains $1978$ names, numbered $1, 2,\ldots$, $1978$. Prove that there is at least one member whose number is the sum of the numbers of two members from his own country, or twice as large as the number of one member from his own country.\hfill(Imo 1978)

	\item Let $A$ and $E$ be opposite vertices of a regular octagon. $A$ frog starts jumping at vertex $A$. From any vertex of the octagon except $E$, it may jump to either of the two adjacent vertices. When it reaches vertex $E$, the frog stops and stays there.. Let a be the number of distinct paths of exactly $n$ jumps ending at $E$. Prove that 
\begin{align}a_2n-1=0, a_{2n} = \frac{1}{\sqrt{2}}\brak{x^{n - 1} - y^{n - 1}}\end{align},
$n = 1, 2, 3 ,\ldots$,
	where $x = 2 + \sqrt{2}$ and $y = 2 - \sqrt{2}$ . Note. A path of a jumps is a sequence of vertices $\brak{P_0\ldots P_n}$ such that
\begin{enumerate}
	\item $PA, P = E$
\item for every $i, 0 \leq i \leq n - 1, P$ is distinct from $E$;
\item for every $i, 0 \leq i \leq n - 1 P$. and $P_{i+1}$ are adjacent.\end{enumerate}\hfill(Imo 1979)
	\subsection*{ Algebra}  
\item Find all real numbers a for which there exist non-negative real numbers $x_1, x_2, x_3, x_4,x_5$ satisfying the relations \begin{align}\sum_{k=1}^{5}kx_{k}=a,\sum{k=1}{5}k^{3}x_{k}=a^2,\sum{k=1}{5}k^{5}x_{k}=a^3\end{align}.\hfill(Imo 1979)
\end{enumerate}
