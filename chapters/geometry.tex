\begin{enumerate}

	\item On each side of an equilateral triangle with side length $n$ units, where $n$ is an integer,$1 \leq n \leq 100$ , consider $n - 1$ points that divide the side into $n$ equal segments.Through these points, draw lines parallel to the sides of the triangle, obtaining a net of equilateral triangles of side length one unit. On each of the vertices of these small triangles, place a coin head up. Two coins are said to be adjacent if the distance between them is $1$ unit. A move consists of flipping over any three mutually adjacent coins. Find the number of values of $n$ for which it is possible to turn all coins tail up after a finite number of moves.\hfill(IOQM 2015)


	\item In an equilateral triangle of side length $6$, pegs are placed at the vertices and also evenly along each side at a distance of $1$ from each other. Four distinct pegs are chosen from the $15$ interior pegs on the sides (that is, the chosen ones are not vertices of the triangle) and each peg is joined to the respective opposite vertex by a line segment.If $N$ denotes the number of ways we can choose the pegs such that the drawn linesegments divide the interior of the triangle into exactly nine regions, find the sum ofthe squares of the digits of $N$.\hfill(IOQM 2015)		
	\item In a triangle $ABC$, let $E$ be the midpoint of $AC$ and $F$ be the midpoint of $AB$. The medians $BE$ and $CF$ intersect at $G$. Let $Y$ and $Z$ be the midpoints of $BE$ and $CF$, respectively. If the area of triangle $ABC$ is 480, find the area of triangle $GYZ$.\hfill(IOQM 2015)
	    \item The six sides of a convex hexagon $A_{1}A_{2}A_{3}A_{4}A_{5}A_{6}$ are colored red. Each of the diagonals of the hexagon is colored either red or blue. If $N$ is the number of colorings such that every triangle $A_{i}A_{j}A_{k}$ , where $1 \leq i < j < k \leq 6$ , has at least one redside, find the sum of the squares of the digits of $N$ .\hfill(IOQM 2015)
    \item Let $X$ be the set of all even positive integers $n$ such that the measure of the angle of some regular polygon is $n$ degrees. Find the number of elements in $X$ .\hfill(IOQM 2015)
    
    \item Let $ABCD$ be a unit square. Suppose $M$ and $N$ are points on $BC$ and $CD$, respectively, such that the perimeter of triangle $MCN$ is $2$. Let $O$ be the circumcenter of triangle $MAN$, and $P$ be the circumcenter of triangle $MON$. If $\left(\frac{OP}{OA}\right)^2 = \frac{m}{n}$ for some relatively prime positive integers $m$ and $n$, find the value of $m + n$.\hfill(IOQM 2015)
    
    \item Let $ABC$ be a triangle in the $xy$-plane, where $B$ is at the origin $\brak{0, 0}$. Let $BC$ be produced to $D$ such that $BC : CD = 1 : 1$, $CA$ be produced to $E$ such that $CA : AE = 1 : 2$, and $AB$ be produced to $F$ such that $AB : BF = 1 : 3$. Let $G\brak{32, 24}$ be the centroid of triangle $ABC$ and $K$ be the centroid of triangle $DEF$. Find the length $GK$.\hfill(IOQM 2015)
    
    \item In the coordinate plane, a point is called a lattice point if both of its coordinates are integers. Let $A$ be the point $\brak{12, 84}$. Find the number of right-angled triangles $ABC$ in the coordinate plane where $B$ and $C$ are lattice points, having a right angle at the vertex $A$ and whose incenter is at the origin $\brak{0, 0}$.\hfill(IOQM 2015)
    
    \item A trapezium in the plane is a quadrilateral in which a pair of opposite sides are parallel. A trapezium is said to be non-degenerate if it has positive area. Find the number of mutually non-congruent, non-degenerate trapeziums whose sides are four distinct integers from the set $\cbrak{5, 6, 7, 8, 9, 10}$.\hfill(IOQM 2015)
    
    \item In triangle  $ABC$, point $ A_1 $ lies on side $ BC $ and point $B_1$ lies on side $ AC $. Let  $P$ and $ Q $ be points on segments 
$AA_1$ and $BB_1 $, respectively, such that   $PQ \parallel AB$.

Let
 $P_1$ be a point on line  $PB_1$ such that $B_1$ lies strictly between 
$P$ and $P_1$, and $\angle PP_1C$ = $\angle BAC$. Similarly, let $Q_1$ 
be a point on line $QA_1$ such that $A_1$ lies strictly between $Q$ and 
$Q_1$, and $ \angle CQ_1Q$ = $\angle CBA $.
Prove that points $P, Q, P_1,$ and $Q_1$ are concyclic.
\hfill(IMO 2019)


\item
 Let $I$ be the in center of acute triangle $ABC$ with $AB$ $\neq AC$. 
The incircle $\omega$ of $ABC$ is tangent to sides $ BC$, $CA$, and $AB$
 at points $D$,  $E$, and $F$, respectively. 


The line 
through $D$ perpendicular to $EF$ meets $\omega$ again at $R$. Line $AR$
 meets $omega$ again at $P$. The circumcircles of triangles $PCE$ and 
$PBF$ meet again at $Q$.

Prove that lines $DI$ and $PQ$ meet on the line through $ A$ that is perpendicular to $AI$.
\hfill(IMO 2019)
\item consider the convex quadrilateral ABCD.The point P is the interior of ABCD.The following ratio equalities hold:
\begin{align}
\angle PAD: \angle PBA: \angle DPA =1:2:3 = \angle CBP: \angle BAP: \angle BPC.
\end{align} 
prove
 that the following three lines meet in a point:the internal bisectors 
of angles $\angle ADP$ and $\angle PCB$ and the perpendicular bisector 
of segment AB
\hfill(IMO 2020)
\item Prove that there exists a positive constant c such that the following statement is true:
Consider an integer n\textgreater1, and a set S of n points in the plane such that the distance between
any
 two different points in S is at least 1. It follows that there is a 
line l separating S such that the distance from any point of S to l is 
at least $cn^\frac{-1}{3}$
(A line l separates a set of points S if some segment joining two points in S crosses l.)
Note. Weaker results with  replaced by $cn^\alpha$ may be awarded points depending on the value of the constant $ \alpha$ \textgreater1/3.
   \hfill(IMO 2020)
\item Let D be an interior point of the acute triangle ABC with AB \textgreater AC so that $\angle DAB = \angle CAD$.The point E on the segment AC satisfies $\angle ADE=\angle BCD$,the point F on the segment AB satisfies $\angle FDA=\angle DBC$,and the point X on the line AC satisfies CX=BX. let $O_{1}$ and $O_{2}$ be the circumcentres of the triangles ADC and EXD,respectively.Prove that the lines BC,EF,and $O_{1} O_{2}$ are concurrent
 \hfill(IMO 2021)
\item Let r be a circle with centre I,and ABCD a convex quadrilateral such that each of the segments AB,BC,CD and DA is a tangent to r.Let \ohm be the circumcircle of the triangle AIC.The extension of BA beyond A meets \ohm at X,and the extension of BC beyond C meets \ohm at Z.The extensions of AD and CD beyond D meet \ohm at Y and T,respectively.Prove that
\begin{align}
    AD+DT+TX+XA=CD+DY+YZ+ZC
\end{align}
 \hfill(IMO 2021)
 \item
Let $ABCDE$  be a convex pentagon such that  $BC = DE$.  Assume that there is a point  $T$  inside  $ABCDE$ with  $TB = TD$,  $TC = TE$  and  $\angle{ABT} = \angle{TEA}$.  Let line  $AB$  intersect 
lines $CD$  and  $CT$  at points  $P$  and $Q$,  respectively. Assume that the points  $P, B, A, Q$  occur on their line in that order. Let line  $AE$  intersect lines  $CD$  and  $DT$  at points  $R$  and  $S$,  respectively. Assume 
that the points $ R, E, A, S $ occur on their line in that order. Prove that the points $ P, S, Q, R $ lie on a circle. \hfill(IM0 2022)
\item
Let  $ABC$ be an acute-angled triangle with  $AB \leq AC$. Let $\Omega$  be the circumcircle of  $ABC$.  Let  $S$ be the midpoint of the arc  $CB$  of  $\Omega$ containing $A$.  The perpendicular from $A$  to  $BC$ meets  $BS$  at  $D$  and meets  $\Omega$  again at$E \neq A$. The line through $
D$ parallel to  $BC$  meets line  $BE$ at  $L$.  Denote the circumcircle of triangle  $BDL$  by  $\omega$.  Let  $\omega$  meet  $\Omega$  again at  $P \neq B$.Prove that the line tangent to $\omega$ at  $P$  meets line $BS$  on the internal angle bisector of  $\angle{BAC}$. \hfill(IMO 2023)
\item
Let  $ABC$  be an equilateral triangle. Let $A_{1}, B_{1}, C_{1}$  be interior points of  $ABC$  such that $ BA_{1} = A_{1}C, CB_{1} = B_{1}A, AC_{1} = C_{1}B,$  and $\angle{BAC} + \angle{CB_{1}A} + \angle{AC_{1}B} = 480^{\circ}.$  Let $ BC_{1}$ and $CB_{1}$  meet at $A_{2}$,  let  $CA_{1}$ and  $A
C_{1}$  meet at  $B_{2}$,  and let  $AB_{1}$ and  $BA_{1}$  meet at  $C_{2}$. Prove that if triangle  $A_{1}B_{1}C_{1}$  is scalene, then the three circumcircles of triangles $AA_{1}A_{2}, BB_{1}B_{2}$  and  $CC_{1}C_{2}$ all pass through two common points.
$\brak{\text{Note: no 2 sides have equal length.}}$ \hfill(IMO 2023)
\item
Let  $ABC$  be a triangle with $ AB \leq AC \leq BC $.  Let the incentre and incircle of triangle   $ABC$  be  $I$ and  $\omega$, respectively. Let $X$ be the point on line  $BC$  different from  $C$  such that the line   through  $X$  parallel to  $AC$  is tangent to  $\omega$.  Similarly, let  $
Y$ be the point on line  $BC$  different from  $B$  such that the line through  $Y$ parallel to  $AB$  is tangent to  $\omega$.  Let $AI$  intersect the circumcircle of  triangle $ABC$  again at  $P \neq A$. Let  $K$  and $L$  be the midpoints of  $AC$  and  $AB$,  respectively.  Prove that  $\angle{KIL} + \angle{YPX} = 180^{\circ}.$ \hfill(IMO 2024)
\item Three points $ X, Y, Z $ are on a straight line such that $ XY = 10 $ and $ XZ = 3 $. What is the product of all possible values of $ YZ $?\hfill(Prermo 2013)

\item Let $ AD $ and $ BC $ be the parallel sides of a trapezium $ ABCD $. Let $ P $ and $ Q $ be the midpoints of the diagonals $ AC $ and $ BD $. If $ AD = 16 $ and $ BC = 20 $, what is the length of $ PQ $?\hfill(Prermo 2013)

\item In a triangle $ ABC $, let $ H $, $ I $, and $ O $ be the orthocenter, incenter, and circumcenter, respectively. If the points $ B $, $ H $, $ I $, and $ C $ lie on a circle, what is the magnitude of $ \angle BOC $ in degrees?\hfill(Prermo 2013)

\item Let $ ABC $ be an equilateral triangle. Let $ P $ and $ S $ be points on $ AB $ and $ AC $, respectively, and let $ Q $ and $ R $ be points on $ BC $ such that $ PQRS $ is a rectangle. If $ PQ = \sqrt{3} \times PS $ and the area of $ PQRS $ is $ \frac{28}{3} $, what is the length of $ PC $?\hfill(Prermo 2013)

\item Let $ A_1, B_1, C_1, D_1 $ be the midpoints of the sides of a convex quadrilateral $ ABCD $ and let $ A_2, B_2, C_2, D_2 $ be the midpoints of the sides of the quadrilateral $ A_1B_1C_1D_1 $. If $ A_2B_2C_2D_2 $ is a rectangle with sides 4 and 6, then what is the product of the lengths of the diagonals of $ ABCD $?\hfill(Prermo 2013)

\item Let $ S $ be a circle with center $ O $. A chord $ AB $, not a diameter, divides $ S $ into two regions $ R_1 $ and $ R_2 $. Let $ S_1 $ be a circle with center in $ R_1 $ touching $ AB $, the circle $ S $ internally. Let $ S_2 $ be a circle with center in $ R_2 $ touching $ AB $ at $ Y $, the circle $ S $ internally, and passing through the center of $ S $. The point $ X $ lies on the diameter passing through the center of $ S_2 $, and $ \angle YXO = 30^\circ $. If the radius of $ S_2 $ is 100, then what is the radius of $ S $?\hfill(Prermo 2013)

\item In a triangle $ ABC $ with $ \angle BCA = 90^\circ $, the perpendicular bisector of $ AB $ intersects segments $ AB $ and $ AC $ at $ X $ and $ Y $, respectively. If the ratio of the area of quadrilateral $ BXYC $ to the area of triangle $ ABC $ is 13:18 and $ BC = 12 $, then what is the length of $ AC $?\hfill(Prermo 2013)
\item $A$ convex hexagon has the property that for a ny pair of opposite sides the distance between their midpoints is $\frac{\sqrt{3}}{2}$ times the sum of their lengths Show that all the hexagon's angles are equal.\hfill(IMO 2003)
\item $ABCD$ is cyclic. The feet of the perpendicula r from $D$ to the lines $AB$, $BC$, $CA$ are $P, Q,R $ respectivel$y$. Show that the angle bisectors of $ ABC$ and $CDA$ meet on the line $AC$ iff $RP = RQ$.\ hfill(IMO 2003)
\item Let $ABC$ be an acute-angled triangle with circumcentre $0$. Let $P$ on $BC$ be the foot o f the altitude from $A$. \\Suppose that $\textless BC S$ $\leq$ $\angle ABC+30^0$. \\Prove that $\textless CAB+\leq cop \angle 90^o$.\hfill(IMO 2001)
\item In a triangle $ABC$, let $AP$ bisect $\angle B AC$, with $P$ on $BC$, and let $BQ$ bisect $\angle A BC$, with $Q$ on $CA$. It is known that $\angle BAC= 60^0$ and that $AB+BP=AQ+QB$. What are the possible angles of triangle $ABC$?\hfill(IMO 2001)
\item $BC$ is a diameter of a circle center $0$. $A$ is any point on the circle with $\angle AOC \textgreater 60^0$. $EF$ is the chord which is the perpendicular bisector of $AO$. $D$ is the midpoint of the minor arc $AB$. The line through $0$ parallel to $AD$ meets $AC$ at $J$. Show that $J$ is the inc enter of triangle $CEF$.\hfill(IMO 2002)
\item $n\textgreater2$ circlesof radius $1$ are drawn in the plane so that no line meets more th an two of the circles. Their centers are $0_{1}, 0_{ 2}\dots0_{n}$. Show that $\sum_{i\textless}$ $1/0_{i }0_{j}\leq \brak{n-1} \frac{\pi}{4}$.\hfill(IMO 2002)
\item In the plane two different points $O$ and $A$ are given.  For each point $X$ of the plane, other than $O$, denote by $a\brak{X}$  the measure of the angle between $OA$ and $OX$ in radians countrclockwise from $OA\brak {O\leq a\brak{X}<2\pi}$. Let $C\brak{X}$ be the circle  with center $O$ and radius of length $\frac {OX+a\brak{X}}{OX}$. each  point  of the plane is colored by one of a finite number of colors. Proveoint $Y$ for which $a\brak{y}>0$ such that color appears on  the circumference of the circle $C\brak{Y}$.\hfill(IMO 1984)
\item Let $ABCD$ be a convex quadrilateral such tha the line $CD$ is a  tangent to the circle on $AB$ as diameter. Prove that the line $AB$ is a tangent to the  circle on $CD$ as diameter if and only if the lines $BC$ and $AD$ are parallel.\hfill(IMO 1984)
\item Let $d$ be the sum of the lengths of all the diagonals of a plane convex polygon with $n$ vertices $\brak{n>3}$,and let $p$ be its perimeter.Prove that.\begin{align*}                                    In-3<\frac{2d}{p}<\myvec{\frac{n}{2}}\myvec{\frac{n+1}{2}}-2,\end{align*}
		Where $\myvec{x}$ denotes the gratest integer not excee    ding $x$  \hfill(IMO 1984)

	
\item let $A$ be one of the two distinct points of intersection of two unequal coplanar tangents to the circles $C_1$ and $C_2$ with centers $ O_1$ and $O_2$, respectively. One of the common tangents to the circles touches $C_1$ at $P_1$ and $C_2$ at $P_2$, while the other touches $C_1$ at $Q_1$ and $C_2$ at $Q_2$.  Let $M_1$ be the midpoint of $P_1Q_1$,$M_2$ be the midpoint of $P_2Q_2$ prove that $\angle O_1AO_2 =\angle M_1AM_2$.\hfill(IMO1983)

     \item $A$ circle has center on the side $AB$ of the cyclic quadrilateral $ABCD$. The other three sides are tangent to the circle. Prove that $AD+BC = AB$.\hfill(IMO 1985)

\item A circle with center $O$ passes through the vertices $A$ and $C$ of triangle $ABC$ and intersects the segments $AB$ and $BC$ again at distinct points $K$ and $N$ respectively. The circumscribed circle of the triangle $ABC$ and $EBN$ intersect at exactly two distinct points $B$ and $M$. Prove that angle $OMB$ is a right angle.\hfill(IMO 1985)
\item $P$ is a point inside a given triangle $ABC.D, E, F$ are the feet of the perpendiculars from $P$ to the lines $BC, CA, AB$ respectively. Find all $P$ for which \\ $\frac{BC}{PD}+\frac{CA}{PE}+\frac{AB}{PF}$ is least. \hfill(IMO 1981)
\item Three congruent circles have a common point $O$ and lie inside     a given triangle. Each circle touches a pair of sides of the triangle. Prove that the incenter and the circumcenter of the triangle and   the point $O$  are collinear\hfill(IMO 1981)     
\item A non-isosceles triangle $A_1 A_2 A_3$ is given with sides $a_1,a_2,a_3$ ($a_i$ is the sid    e opposite $A_i$). For all $i = 1, 2, 3, M_i$ is the midpoint of side $a_i$ and $T_i$ is the point where     the incircle touches side $a_i$. Denote by $S_i$ the reflection. of $T_i$ in the interior bisector of anngle $A_i$. Prove that the lines $M_1,S_1$,$ M_2S_2$ and $M_3S_3$ are concurrent.\hfill(IMO 1982)
 \item The diagonals $AC$ and $CE$ of the regular hexagon $ABCDEF$ are divided by the inner points $M$ and $N$, respectively, so that \begin{align*} \frac{AM}{AC}=\frac{CN}{CE}=r.
                   \end{align*}
 Determine r if $B,$ $M,$ and $N$ are collinear. \hfill(IMO 1982)
 \item Let $S$ be a square with sides of length $100$, and let $L$ be a path with in $S$ which does not meet itself and which is composed of line segments $A_0A_1, A_1A_2,.... A_{n-1}A_1$ with $A_0 \neq A_n$.     Suppose that for every point $P$ of the boundary of $S$ there is a point of $L$ at a distance from $P$ not greater than $\frac{1}{2}$. Prove that there are two points $X$ and $Y$ in  $\&$ such that the distance between $X$ and $Y$ is not greater than $1$, and the length of that part of $L$ which lies between $X$ and $Y$ is not smaller than $198$.\hfill(IMO 1982)
\item A triangle $A_1A_2A_3$ and a point $P_0 $are given in the plane.We define $A_s
    =A_s-3$ for all $s\geq4$ .We construct a set of points $P_1$, $P_2$,$P_3$\dots,such that $P_{k+1}$ is the image of $P_k$ under a rotation with center $A_{k+1}$ through angle $120^\circ$ clockwise $\brak{for \space k=0,1,2,3\dots}$ .Prove that if $P_{1986}$=$P_0$, then thetriangle $A_1A_2A_3$ is equilateral .\hfill(IMO 1986)

    \item Let $A$, $B$ be adjacent vertices of a regular n-gon $\brak{n\leq5}$ in the plane having center at $O$. A triangle $XYZ$, which is congruent to and initially conincides with $OAB$, moves in the plane in such a way that $Y$ and $Z$ each trace out the whole boundary of the polygon, $X$ remaining inside the polygon. Find the locus of $X$.\hfill(IMO 1986)

    \item In an acute-angled triangle $ABC$ the interior bisector of the angle $A$ intersects $BC$ at $L$ and intersects the circumcircle of $ABC$ again at $N$. From point $L$ perpendiculars are drawn to $AB$ and $AC$, the feet of these perpendiculars being $K$ and $M$respectively. Prove that the quadrilateral $AKNM$ and the triangle $ABC$ have equal areas.\hfill(IMO 1987)

    \item Prove that there is no function $f$ from the set of non-negative integers into itself such that $f\brak{f\brak{n}}=n+1987$ for every $n$.\hfill(IMO 1987)

    \item Consider two coplanar circles of radii $R$ and $r$ $\brak{R > r}$ with the same center. Let $P$ be a fixed point on the smaller circle and $B$ a variable point on the lar ger circle. The line $BP$ meets the larger circle again at $C$. The perpendicular $l$ to $BP $ at $P$ meets the smaller circle again at $A$. (If $l$ is tangent to the circle at $P$ then $A = P$)
                 $\brak{i}$ Find the set of values of $BC^2+CA^2+AB^2$ 
                 $\brak{ii}$ Find the locus of the midpoint of $BC$.\hfill(IMO 1988)

\item $ABC$ is a triangle right-angled at $A$, and $D$ is the foot of the altitude from $A$. The straight line joining the incenters of the triangles $ABD$, $ACD$ intersects the sides $AB$, $AC$ at the points $K$, $L$ respectively. $S$ and $T$ denote the areas of the triangles $ABC$ and $AKL$ respectively. Show that $S\geq 2T$.\hfill(IMO 1988)
\item Problem $5$. A configuration of $4027$ points in the plane is called Colombian if it consists of $2013$ red points and $2014$ blue points, and no three of the points of the configuration are collinear. By drawing some lines, the plane is divided into several regions. An arrangement of lines is good for a Colombian configuration if the following two conditions are satisfied:
 * no line passes through any point of the configuration;
* no region contains points of both colours

Find the least value of $k$ such that for any Colombian configuration of $4027$ points, there is a good arrangement of $k$ lines \hfill(Imo 2013)
\item Problem 6. Let the excircle of triangle $ABC$ opposite the vertex $A$ be tangent to the side $BC$ at the point $A_1$. Define the points $B_1$, on $CA$ and $C_1$, on $AB$ analogously, using the excircles opposite $B$ and $C$. respectively. Suppose that the circumcentre of triangle $A_1B_1C_1$, lies on the circumcircle of triangle $ABC$. Prove that triangle $ABC$ is right-angled. \hfill(Imo 2013)
                        
	The excircle of triangle $ABC$ opposite the vertex $A$ is the circle that is tangent to the line segment $BC$, to the ray $AB$ beyond $B$, and to the ray $AC$ beyond $C$. The excircles opposite $B$ and $C$ are similarly defined. \hfill(Imo 2013)
\item problem7 Let $ABC$ be an acute-angled triangle with orthocentre$ H$, and let $W$ be a point on the side $BC$, lying strictly between $B$ and $C$. The points $M$ and $N$ are the fect of the altitudes from $B$ and $C$, respectively. Denote by $w_1$ the circumcircle of $BWN$, and let $X$ be the point on wy such that $WX$ is a diameter of $w_1$ Analogously, denote by $w_2$ the circumcircle of $CWM$. and let $Y$ be the point on such that $WY$ is a diameter of Prove that $X$, $Y$ and Hare collinear. \hfill(Imo 2013)
	
\item Problem 8. Let $ Q_{>0}$ be the set of positive rational mumbers. Let $f: Q_{>0} \rightarrow R$ be a function satisfying the following three conditions:
	\begin{enumerate}
		\item for all $x,y\epsilon  Q>0$, we have $f\brak{x} f\brak{y} \geq  f\brak{xy}$
		\item for all $x,y\epsilon Q>0$,we have$f\brak{x+y} \geq f\brak{x}+f\brak{y}$
		\item there exists a rational number $a> 1$ such that $f\brak{a}=a$.
			
		prove that $F\brak{x}=x$ for all $x \epsilon Q>0$.
	\end{enumerate} \hfill(Imo 2013)			
\item Problem 9. let $n\geq 2$ be an integer. Consider an $n\times n$ chessboard consisting of $n^2$ unit squares. A configuration of $n$ rooks on this board is peaceful if every row and every column contains exactly one rook. Find the greatest positive integer $k$ such that, for each peaceful configuration of $n$ rooks, there is a $k\times k$ square which does not contain a rook on any of its $k^2$ unit squares. \hfill(Imo 2014)
	
\item Problem 10. Convex quadrilateral $ABCD$ has $\angle ABC= \angle CDA = 90 \degree$ Point His the foot of the perpendicular from A to BD. Points S and T lie on sides $AB and AD$, respectively, such that $H$ lies inside triangle $SCT$ and $\angle CHS- \angle CSB = 90 \degree , \angle THC- \angle DTC = 90\degree$ .
	Prove that line $BD$ is tangent to the circumcircle of triangle $TSH$. \hfill(Imo 2014)
\item Problem 4. Points $P and Q$lie on side $BC$ of acute-angled triangle $ABC$ so that $\angle PAB= \angle BCA$ and $\angle CAQ=\angle ABC.$ Points $M$ and $N$ lie on lines $AP$ and $AQ,$ respectively, such that $P$ is the midpoint of $AM,$ and $Q$ is the midpoint of $AN.$ Prove that lines $BM and CN$ intersect on circumcircle of triangle $ABC$ \hfill(Imo 2014)
\item Problem 11. A set of lines in the plane is in general position if no two are parallel and no three pass through the same point. A set of lines in general position cats the plane into regions, some of which have finite area; we call these its finite regions. Prove that for all sufficiently large $n$. in any set of a lines in general position it is possible to colour at least $\sqrt n$ of the lines blue in such a way that none of its finite regions has a completely blue boundary.

	Note: Results with $\sqrt n$ replaced by $c \sqrt n$  will be awarded points depending on the value of the constant $c$. \hfill(Imo 2014)
	
\item Problem 12. We say that a finite set $S$ of points in the plane is balanced if, for any two different points $A and B$ in $S$, there is a point Cin Ssuch that $AC=BC$. We say that $S$ is centre-free if for any three different points $A, B$ and $C$ in $S$, there is no point $P$ in $S$ such that $PA=PB=PC$
	\begin{enumerate} 

\item  Show that for all integers $n\geq3$, there exists a balanced set consisting of $n$ points.

\item  Determine all integers $n\geq3$ for which there exists a balanced centre-free set consisting of $n$ points.
	\end{enumerate}	\hfill(Imo 2015)	
	
\item Problem 13. Determine all triples $\brak{a, b, c}$ of positive integers such that each of the numbers
			 $ ab-c, bc-a,ca-b$\\
		is a  power of $2$\\(A power of 2 is an integer of the form $2^n$,Where $n$ is a non-negative integer). \hfill(Imo 2015)
		
\item Problem 14. Let $ABC$ be an acute triangle with $AB\textgreater AC$ Let I be its circumcircle, $H$ its orthocentre, and $F$ the foot of the altitude from $A$. Let $M$ be the midpoint of $BC$. Let $Q$ he the point on $T$ such that $\angle HQA= 90$, and let $K$ be the point on $T$ such that $\angle HKQ=90\degree.$ Assume that the points$ A, B, C, K and Q$ are all different, and lie on $T$ in this order.

	Prove that the circumcircles of triangles $KQH$ and $FKM$ are tangent to each other. \hfill(Imo2015)
\item Problem 15. Triangle $ABC$ has circumcircle $\ohm$ and circumcentre $O$. A circle $T$ with centre. A intersects the segment $BC$ at points $D and E$, such that $B, D, E $and Care all different and lie on line $BC$ in this onter. Let $F and G $be the points of intersection of $T and \ohm$. such that $A. F B. C and G $lie on \ohm in this order. Let $K $ he the second point of intersection of the circumcircle of triangle $BDF$ and the segment $AB$. Let $L$ be the second point of intersection of the circumcircle of triangle $CGE$ and the segment $CA$
	Suppose that the lines $FKand GL$ are different and intersect at the point $X$. Prove that $X$ lies on the line $AO$. \hfill(Imo 2015)
\item Problem 16. Let $R$ be the set of real numbers. Determine all functions $f:R\rightarrow R$ satisfying the equation
	\begin{align}
		f\brak{x+f\brak{x+y}}+f\brak{xy}=x+f\brak{x+y}+yf\brak{x}
	\end{align}
for all real numbers $x$ and $y$ \hfill(Imo2015)

\item problem17 the sequence $a_1,a_2, \ldots$ of an integers satisfies the following conditions;
	\begin{enumerate}
		\item $1\leq a_{j} \leq2015$ for all $j\geq 1;$
		\item $k+a_{k} \neq l+a_{l}$ for all $1\leq k \textless l.$
	\end{enumerate}	
prove that there exist two positive integers $b and N$ such that

$\mydet {\sum_{j=m+1}^{n} \brak {aj-b} }\leq 1007^2$

for all integers $m and n$ satisfying $n > m\geq N$ \hfill(Imo 2015)
\item Prove that the set $\cbrak{1,2,.........,1989}$ can be expressed as the disjoint union of subsets $A_i$\brak{i=1,2,........,117} such that :
\brak{i} Each $A_i$ contains $17$ elements ;
		\brak{ii} The sum of all the elements in each $A_i$ is the same . \hfill(IMO 1989)


\item In an acute-angled triangle $ABC$ the internal bisector of angle $A$ meets the circumcircle of the triangle again at $A_1$. Points $B_1$ and $C_1$ are defined similarly. Let $A_0$ be the point of intersection of the line $AA_1$ with the external bisectors of angles $B$ and $C$. Points $B_0$ and $C_0$ are defined similarly. Prove that: 

\brak{i} The area of the triangle $A_0$ $B_0C_0$ is twice the area of the hexagon $AC_1BA_1CB_1$

\brak{ii} The area of the triangle $A_0B_0C_0$ is at least four times the area of the triangle $ABC$. \hfill(IMO 1989)

\item Let $n$ and $k$ be positive integers and let $S$ be a set of $n$ points in the plane such that

\brak{i} No three points of $S$ are collinear, and 

\brak{ii} For any point $P$ of $S$ there are at least $k$ points of $S$ equidistant from $P$. \hfill(IMO 1989)

		Prove that: \begin{align*}k < \frac{1}{2} + \sqrt{2n}.\end{align*}

	\item Let $ABCD$ be a convex quadrilateral such that the sides ${AB, AD, BC}$ satisfy $AB= AD + BC$. There exists a point. $P$ inside the quadrilateral at a distance $h$ from the line $CD$ such that $AP= h+ AD$ and $BP= h + BC$. Show that:\begin{align*}
	\frac{1}{\sqrt{h}}\geq\frac{1}{\sqrt{AD}}+\frac{1}{\sqrt{BC}}\end{align*}. \hfill(IMO 1989)


   \item Chords $AB$ and $CD$ of a circle imersect at a point $E$ inside the circle. Let $M$ be an interior point of the segment $EB$. The tangen    t line at $E$ to the circle through $D, E$. and $M$ intersects the lines $BC$ and $AC$ at $F$ and $G$. respectively,                                      If \begin{align*}\frac{AM}{AB}=t \end{align*}      
  find \begin{align*} \frac{EG}{EF}\end{align*}  
	  in terms of t .\hfill(IMO 1990)


      \item Let $n_3$ and consider a set $E$ of $2_{n-1}$ distinct points on a circle. Suppose that exactly $k$ of these points are to he colored black. Such a coloring is $"good"$ if there is at least  one pair of black points such that the interior of one of the ares between them contains exactly in points from $E$. Find the smallest value of $k$ so that every such coloring of $k$ points of $E$ is good \hfill(IMO 1990)


     \item Given an initial integer $n_0 > 1$, two players. $A$ a    nd $B$, choose integers $n_1, n_2 , n_3,.......$ alternately accordi    ng to the following rules:
           Knowing $n_{2k}$, $A$ chooses any integer $n_{2k+2}$ such that \begin{align*} n_{2k}\leq n_{2k+1} \leq n^{2}_2{k} \end{align*}
   Knowing $n_{2k+1}$ , $B$ chooses any integer $n_{2k+2}$ such that \begin{align*}
              \frac{n_{2k+1}}{n_{2k+2}}\end{align*}
  is a prime raised to a positive integer power.
	Plaver $A$ wins the game by choosing the number $1990$: player $B$ wins by choosing the number $1$. For which $n_0$ does:
\brak{a}$A$ have a winning strategy?
\brak{b} $B$ have a winning strategy?
\brak{c} Neither player have a winning strategy?\hfill(IMO 1990)

\item Prove that there exists a convex $1990$-gon with the following     two properties
\brak{a} All angles are equal.
\brak{b} The lengths of the 1990 sides are the numbers $1^2, 2^2, 3^    2$,.....,$1990^2$ in some order.\hfill(IMO 1990)

\item Let $ABC$ be a triangle and $P$ an interior point of $ABC$.     Show that at least one of the angles $\angle{PAB}, \angle{PBC}, \angle{PCA}$ is less than or equal to $30\degree$.\hfill(IMO 1991)


\end{enumerate}
